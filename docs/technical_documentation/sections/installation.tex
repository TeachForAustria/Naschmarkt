\section{Installation vorbereiten}
Bevor die Applikation auf das Zielsystem ausgeliefert werden kann, muss der Naschmarkt zur Installation bereit gemacht werden.
In diesem Schritt werden Abh\"angigkeiten installiert und verschiedenste Buildprozesse ausgef\"uhrt.
Dieser Schritt kann sowohl auf dem lokalen Host also auch dem Zielserver erfolgen.

Daf\"ur werden aus dem Projektverzeichnis folgende Kommandos ausgef\"uhrt:
\begin{lstlisting}[caption=Installationsvorbereitung]
# Install composer (PHP) dependencies
composer install

# Install frontend and JavaScript dependencies
npm install

# Run build tasks
gulp
\end{lstlisting}

\section{Konfiguration}
Um den Naschmarkt zu konfiguration muss eine Umgebunskonfiguration (Environment) erstellt werden.
F\"ur diese Konfigurationsdatei gibt es im Hauptverzeichnis ein Beispiel namens \texttt{.env.example}.

\begin{lstlisting}[caption=Installationsvorbereitung]
# no need to change this
APP_ENV=local
APP_DEBUG=true
APP_KEY=SomeRandomString

# database configuration
DB_HOST=127.0.0.1
DB_DATABASE=homestead
DB_USERNAME=homestead
DB_PASSWORD=secret

# no need to change this
CACHE_DRIVER=file
SESSION_DRIVER=file
QUEUE_DRIVER=sync

# Mail configuration
# See https://laravel.com/docs/5.2/mail#introduction for details
MAIL_DRIVER=smtp
MAIL_HOST=mailtrap.io
MAIL_PORT=2525
MAIL_USERNAME=null
MAIL_PASSWORD=null
MAIL_ENCRYPTION=null
\end{lstlisting}
Diese wird nach \texttt{.env} kopiert und entsprechend bearbeitet.
Weite Konfigurationsparameter befinden sich in den Dateien des \texttt{config} Ordners, diese sollten unter normalen Umst\"anden jedoch keiner Anpassung bed\"urfen.
F\"ur eine genauere Beschreibung sei an die Konfigurationsdokumentation unter \url{https://laravel.com/docs/5.2/configuration} von Laravel verwiesen.
Danach muss noch ein Application Key generriert werden, dies erfolgt mit folgendem Befehl:

\begin{lstlisting}[caption=Installationsvorbereitung]
php artisan key:generate
\end{lstlisting}

Je nachdem, wo das Vorbereitung erfolgt ist, muss sp\"atenstens jetzt der Naschmarkt auf den Server hochgeladen werden.
Der Webroot des Webserver soll dabei auf der \texttt{public} Verzeichnis zeigen.
