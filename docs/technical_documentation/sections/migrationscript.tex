Die Migration wird \"uber 2 in Python 2.7 geschriebenes Skripts abgehaldelt, welches die aus der aktuellen Wordpress Datenbank nimmt und sie auf die neue Datenstruktur umwandelt.
Zuerst sollte das Skript, welches die Benutzer migriert ausgef\"uhrt werden und dann das Skript welche alle Posts migriert.

\section{Ben\"otigte Bibliotheken}
Um das Skript ausf\"uhren zu k\"onnen, m\"ussen folgende Python Bibliotheken am ausf\"uhrenden Computer installiert sein.

\begin{itemize}
  \item \href{http://mysql-python.sourceforge.net/MySQLdb.html}{ \texttt{MySQLdb} }, welche jeweils zum Holen und Schreiben der Daten von der MySQL Datenbank verwendet wird
  \item \href{https://textract.readthedocs.io/en/latest/}{ \texttt{textract} }, welches f\"ur die Volltext-Suche die Datein indexiert
\end{itemize}

Das Skript wird dann \"uber die Kommandozeile mit \texttt{python} oder \texttt{python2.7 migrateUsers} oder \texttt{migrate Posts}

\section{Konfigurations Datei}
Das Skript kommt mit einer Konfigurationsdatei \texttt{config.py}, in welcher man die ben\"otigten Daten f\"ur

\begin{itemize}
  \item die alte Datenbank, woher die Daten kommen
  \item die neue Datenbank, wohin die Daten gespeichert werden
  \item den FTP Zugang um die Datein zu speichern\\
\end{itemize}

Hier ein Beispiel f\"ur das config file
\\
\begin{lstlisting}[language={Python}, caption=config.py]
"""

This is the config file for the Migration

There are 3 things to configure.
    - the old Database to migrate from
    - the new Database to save the migration
    - FTP connection to save the files

"""

# Old Database.
# This is where the Data is taken from
dbOld = {
    'host':         "mysql003.tophosting.at",   # host ip
    'port':         3306,                       # port
    'user':         "dbue88a3f4",               # username
    'password':     "Khy?8k9p",                 # password
    'database':     "dbf26d5267"                # name of
                                                # the database
}

# New Database.
# This is where the Data will be stored
dbNew = {
    'host':         "127.0.0.1",                # host ip
    'port':         33060,                      # port
    'user':         "homestead",                # username
    'password':     "secret",                   # password
    'database':     "homestead"                 # name of
                                                # the database
}

# FTP connection to save the files
ftpConnection = {
    'host':         "192.168.10.10",            # host ip
    'user':         "vagrant",                  # username
    'password':     "vagrant",                  # password
    'directory':    "naschmarkt/storage/app"    # directory where
                                                # to save the
                                                # files to
}
\end{lstlisting}
