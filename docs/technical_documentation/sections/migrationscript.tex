Die Migration der Daten vom alten Naschmarkt zu der neuen Applikation erfolgt \"uber zwei in Python 2.7 geschriebenes Skripts, welches die Daten aus der aktuellen Wordpress Datenbank auslesen, in die neue Datenstruktur umwandelt und speichert.
Zuerst wird das Skript zur Migration der Benutzer ausgef\"uhrt, danach folgt der Import der Posts.

\section{Ben\"otigte Bibliotheken}
Um das Skript ausf\"uhren zu k\"onnen, m\"ussen folgende Python Bibliotheken am ausf\"uhrenden Computer installiert sein.

\begin{itemize}
  \item \href{http://mysql-python.sourceforge.net/MySQLdb.html}{\texttt{MySQLdb} (http://mysql-python.sourceforge.net/MySQLdb.html)}: Zum Herunterladen und Schreiben der Daten von der MySQL Datenbank
  \item \href{https://textract.readthedocs.io/en/latest/}{ \texttt{textract} (https://textract.readthedocs.io/en/latest/)}: Indizierung der Dateien
\end{itemize}

Die Ausf\"uhrung erfolgt durch Eingabe folgendes Befehls auf der Kommandozeile:
\begin{lstlisting}[caption=Migration vom alten Naschmarkt]
# migrate users
python2.7 migrateUsers.py

# migrate posts
python2.7 migratePosts.py
\end{lstlisting}

\section{Konfigurationsdatei}
Das Skript kommt mit einer Konfigurationsdatei \texttt{config.py}, die folgende Konfigurationsparameter enth\"alt

\begin{itemize}
  \item Verbindungsinformationen der alten Datenbank
  \item Verbindungsinformationen der neuen Datenbank
  \item  FTP Zugang um die Datein zu speichern
\end{itemize}

Hier ein Beispiel f\"ur dieser Konfigurationsdatei:
\begin{lstlisting}[language={Python}, caption=config.py]
"""

This is the config file for the Migration

There are 3 things to configure.
    - the old Database to migrate from
    - the new Database to save the migration
    - FTP connection to save the files

"""

# Old Database.
# This is where the Data is taken from
dbOld = {
    'host':         "old.host",   # host ip
    'port':         3306,                       # port
    'user':         "username",               # username
    'password':     "password",                 # password
    'database':     "old_naschmarkt"                # name of
                                                # the database
}

# New Database.
# This is where the Data will be stored
dbNew = {
    'host':         "127.0.0.1",                # host ip
    'port':         33060,                      # port
    'user':         "homestead",                # username
    'password':     "secret",                   # password
    'database':     "homestead"                 # name of
                                                # the database
}

# FTP connection to save the files
ftpConnection = {
    'host':         "192.168.10.10",            # host ip
    'user':         "vagrant",                  # username
    'password':     "vagrant",                  # password
    'directory':    "naschmarkt/storage/app"    # directory where
                                                # to save the
                                                # files to
}
\end{lstlisting}
